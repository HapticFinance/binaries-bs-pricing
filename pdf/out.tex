% Options for packages loaded elsewhere
\PassOptionsToPackage{unicode}{hyperref}
\PassOptionsToPackage{hyphens}{url}
%
\documentclass[
]{article}
\usepackage{amsmath,amssymb}
\usepackage{lmodern}
\usepackage{iftex}
\ifPDFTeX
  \usepackage[T1]{fontenc}
  \usepackage[utf8]{inputenc}
  \usepackage{textcomp} % provide euro and other symbols
\else % if luatex or xetex
  \usepackage{unicode-math}
  \defaultfontfeatures{Scale=MatchLowercase}
  \defaultfontfeatures[\rmfamily]{Ligatures=TeX,Scale=1}
\fi
% Use upquote if available, for straight quotes in verbatim environments
\IfFileExists{upquote.sty}{\usepackage{upquote}}{}
\IfFileExists{microtype.sty}{% use microtype if available
  \usepackage[]{microtype}
  \UseMicrotypeSet[protrusion]{basicmath} % disable protrusion for tt fonts
}{}
\makeatletter
\@ifundefined{KOMAClassName}{% if non-KOMA class
  \IfFileExists{parskip.sty}{%
    \usepackage{parskip}
  }{% else
    \setlength{\parindent}{0pt}
    \setlength{\parskip}{6pt plus 2pt minus 1pt}}
}{% if KOMA class
  \KOMAoptions{parskip=half}}
\makeatother
\usepackage{xcolor}
\usepackage[margin=35mm]{geometry}
\usepackage{longtable,booktabs,array}
\usepackage{calc} % for calculating minipage widths
% Correct order of tables after \paragraph or \subparagraph
\usepackage{etoolbox}
\makeatletter
\patchcmd\longtable{\par}{\if@noskipsec\mbox{}\fi\par}{}{}
\makeatother
% Allow footnotes in longtable head/foot
\IfFileExists{footnotehyper.sty}{\usepackage{footnotehyper}}{\usepackage{footnote}}
\makesavenoteenv{longtable}
\setlength{\emergencystretch}{3em} % prevent overfull lines
\providecommand{\tightlist}{%
  \setlength{\itemsep}{0pt}\setlength{\parskip}{0pt}}
\setcounter{secnumdepth}{-\maxdimen} % remove section numbering
\usepackage{float}
\usepackage{pdfpages}
\usepackage{tabu}
\usepackage{lipsum}
\usepackage{booktabs}
\usepackage[justification=raggedright,labelfont=bf,singlelinecheck=false]{caption}
\usepackage{array}
\usepackage{xcolor}
\usepackage{color, colortbl}
\usepackage{amsmath}
\usepackage{mathtools,mathptmx}
\usepackage{bbm}
\usepackage{tabularx}
\usepackage{background}
\usepackage[english]{babel}
\usepackage{csquotes}
\usepackage[style=alphabetic, backend=bibtex]{biblatex}
\bibliography{bibliography/haptic.bib}
\usepackage{tikz}
\usepackage[font=small,labelfont=bf]{caption}
\usetikzlibrary{shapes,positioning}
\usepackage{wrapfig}
\usepackage{eso-pic,graphicx,transparent}
\DeclareUnicodeCharacter{2212}{-}
\backgroundsetup{pages={some},contents={}, opacity={0.3}, color={gray}}
\ifLuaTeX
  \usepackage{selnolig}  % disable illegal ligatures
\fi
\IfFileExists{bookmark.sty}{\usepackage{bookmark}}{\usepackage{hyperref}}
\IfFileExists{xurl.sty}{\usepackage{xurl}}{} % add URL line breaks if available
\urlstyle{same} % disable monospaced font for URLs
\hypersetup{
  pdftitle={Binary Option Valuation},
  hidelinks,
  pdfcreator={LaTeX via pandoc}}

\title{Binary Option Valuation}
\author{}
\date{\vspace{-2.5em}}

\begin{document}
\maketitle

\captionsetup[table]{labelformat=empty}
\captionsetup[table]{labelfont=bf}

\section{Risk-neutral Valuation}

We assume the asset price follows a lognormal random walk in a
frictionless, time continuum setting. The expected return equals the
risk-free interest rate which is taken to be constant and continuous
over time. In addition, the asset does not pay dividends over the life
of the option. This method is called ``risk-neutral'' valuation
approach.

\subsection{Asset-or-nothing Call}

With this type of option, the payout is one unit of the underlying asset
if the spot price \(S\) at maturity \(T\) (\(S_{(T)}\)) is above the
strike price \(K\) or zero otherwise.

\begin{equation*}
V_{AC_{(T)}} = v_{AC}(T, S_{(T)}) = S_{(T)} \Pi_{[S_{(T)} > K]} 
\end{equation*}

Given that the expected return is the risk-free interest rate \(r\), we
get:

\begin{flalign}
    V_{AC} = {\rm e}^{-rT} & \underbrace{ \mathbb{E}[S_{(T)}\Pi_{\{S_{(T)} > K\}}]
  }_\text{Partial expectation} \nonumber \\ 
  & \mathbb{E}[S_{(T)}\Pi_{\{S_{(T)} > K\}}] = S{\rm e}^{rT} \Phi(d_{1}) \nonumber \\ 
  \vspace{1cm}
  & \mathnormal{\:\:\:\:\:\:\:\:\:where} \nonumber \\
  & d_{1} = \frac{1}{\sigma\sqrt{T}} \Big [\ln\Big(\frac{S}{K}\Big)+\Big(r + \frac{1}{2} \sigma^2\Big)T \Big] \nonumber \\
  \nonumber \\
  V_{AC} = {\rm e}^{-rT} & S{\rm e}^{rT} \Phi(d_{1}) = S{\rm e}^{-rT} \Phi(d_{1}) \label{eq:1} 
\end{flalign}

Multiplying \(\Phi(d_{1})\) by the current asset price and the risk-free
compounding factor gives the expected value of receiving the asset at
expiration of the option - contingent upon the contract ending up in the
money - calculated using risk-adjusted probabilities. Therefore,
\(\Phi(d_{1})\) is a measure by which the discounted expected value of
contingent receipt of the asset exceeds the present value of the asset.

\subsection{Asset-or-nothing Put}

With this type of option, the payout is one unit of the underlying asset
if the spot price S at maturity \(T\) is below the strike price K or
zero otherwise.

\begin{equation*}
    V_{AP_{(T)}} = v_{AP}(T, S_{(T)}) = S_{(T)}\Pi_{\{S_{(T)} < K\}} 
\end{equation*}

\newpage

Using put-call parity, we can derive the value of an asset-or-nothing
binary put option (\(V_{AP}\)) by substituing the value of \(V_{AC}\) in
the formula for \(V_{AP}\):

\begin{flalign*}
  V_{AP} & = S {\rm e}^{-rT} - V_{AC} \\
  & = S{\rm e}^{-rT}-S{\rm e}^{-rT} \Phi(d_{1}) \\
  & = S{\rm e}^{−rT} (1 − \Phi(d_{1})) \\
  & = S{\rm e}^{-rT} \Phi(-d_{1})
 \end{flalign*}

\subsection{Cash-or-nothing Call}

With this type of option, the payout is one unit of riskless asset if
the spot price S at maturity \(T\) is above the strike price K or zero
otherwise.

\begin{equation*}
    V_{CC_{(T)}} = v_{CC} (T, S_{(T)}) = \Pi_{[S_{(T)}>K]} 
\end{equation*}

Given that the expected return is the risk-free interest rate \(r\), we
get:

\begin{flalign}
 V_{CC} & = {\rm e}^{-rT} \mathbb{E}[v_{CC}(T, S_{(T)})] \nonumber \\
 & = {\rm e}^{-rT} \mathbb{E}[\Pi_{[S_{(T)}>K]}] \nonumber \\
    & = {\rm e}^{-rT} \underbrace{\mathbb{P} [S_{(T)}>K]}_\text{Tail probability} \nonumber \\ 
  & \mathbb{P}[S_{(T)} > K] = \Phi(d_{2}) \nonumber \\ 
  & \mathnormal{\:\:\:\:\:\:\:\:\:where} \nonumber \\
  & d_{2} = \frac{1}{\sigma\sqrt{T}}\Big[\ln\Big(\frac{S}{K}\Big) + \Big(r - \frac{1}{2} \sigma^2\Big)T\Big] \nonumber \\
  \nonumber \\
  V_{CC} & = {\rm e}^{-rT} \Phi(d_{2}) \label{eq:2} 
\end{flalign}

The value of a this option equals the risk-free compounding factor
multiplied by \(\Phi{(d_{2})}\), the risk-adjusted probability that the
option will be exercised.

\newpage

\printbibliography

\end{document}
